\documentclass[journal,12pt,twocolumn]{IEEEtran}

\usepackage{setspace}
\usepackage{gensymb}
\singlespacing
\usepackage{amsmath}
\usepackage{amsthm}

\usepackage{mathrsfs}
\usepackage{txfonts}
\usepackage{stfloats}
\usepackage{bm}
\usepackage{cite}
\usepackage{cases}
\usepackage{subfig}

\usepackage{longtable}
\usepackage{multirow}

\usepackage{enumitem}
\usepackage{mathtools}
\usepackage{steinmetz}
\usepackage{tikz}
\usepackage{circuitikz}
\usepackage{verbatim}
\usepackage{tfrupee}
\usepackage[breaklinks=true]{hyperref}
\usepackage{graphicx}
\usepackage{tkz-euclide}

\usetikzlibrary{calc,math}
\usepackage{listings}
    \usepackage{color}                                            %%
    \usepackage{array}                                            %%
    \usepackage{longtable}                                        %%
    \usepackage{calc}                                             %%
    \usepackage{multirow}                                         %%
    \usepackage{hhline}                                           %%
    \usepackage{ifthen}                                           %%
    \usepackage{lscape}     
\usepackage{multicol}
\usepackage{chngcntr}

\DeclareMathOperator*{\Res}{Res}

\renewcommand\thesection{\arabic{section}}
\renewcommand\thesubsection{\thesection.\arabic{subsection}}
\renewcommand\thesubsubsection{\thesubsection.\arabic{subsubsection}}

\renewcommand\thesectiondis{\arabic{section}}
\renewcommand\thesubsectiondis{\thesectiondis.\arabic{subsection}}
\renewcommand\thesubsubsectiondis{\thesubsectiondis.\arabic{subsubsection}}


\hyphenation{op-tical net-works semi-conduc-tor}
\def\inputGnumericTable{}                                 %%

\lstset{
%language=C,
frame=single, 
breaklines=true,
columns=fullflexible
}

\begin{document}

\newcommand{\BEQA}{\begin{eqnarray}}
\newcommand{\EEQA}{\end{eqnarray}}
\newcommand{\define}{\stackrel{\triangle}{=}}
\bibliographystyle{IEEEtran}
\raggedbottom
\setlength{\parindent}{0pt}
\providecommand{\mbf}{\mathbf}
\providecommand{\pr}[1]{\ensuremath{\Pr\left(#1\right)}}
\providecommand{\qfunc}[1]{\ensuremath{Q\left(#1\right)}}
\providecommand{\sbrak}[1]{\ensuremath{{}\left[#1\right]}}
\providecommand{\lsbrak}[1]{\ensuremath{{}\left[#1\right.}}
\providecommand{\rsbrak}[1]{\ensuremath{{}\left.#1\right]}}
\providecommand{\brak}[1]{\ensuremath{\left(#1\right)}}
\providecommand{\lbrak}[1]{\ensuremath{\left(#1\right.}}
\providecommand{\rbrak}[1]{\ensuremath{\left.#1\right)}}
\providecommand{\cbrak}[1]{\ensuremath{\left\{#1\right\}}}
\providecommand{\lcbrak}[1]{\ensuremath{\left\{#1\right.}}
\providecommand{\rcbrak}[1]{\ensuremath{\left.#1\right\}}}
\theoremstyle{remark}
\newtheorem{rem}{Remark}
\newcommand{\sgn}{\mathop{\mathrm{sgn}}}
\providecommand{\abs}[1]{\vert#1\vert}
\providecommand{\res}[1]{\Res\displaylimits_{#1}} 
\providecommand{\norm}[1]{\lVert#1\rVert}
%\providecommand{\norm}[1]{\lVert#1\rVert}
\providecommand{\mtx}[1]{\mathbf{#1}}
\providecommand{\mean}[1]{E[ #1 ]}
\providecommand{\fourier}{\overset{\mathcal{F}}{ \rightleftharpoons}}
%\providecommand{\hilbert}{\overset{\mathcal{H}}{ \rightleftharpoons}}
\providecommand{\system}{\overset{\mathcal{H}}{ \longleftrightarrow}}
	%\newcommand{\solution}[2]{\textbf{Solution:}{#1}}
\newcommand{\solution}{\noindent \textbf{Solution: }}
\newcommand{\cosec}{\,\text{cosec}\,}
\providecommand{\dec}[2]{\ensuremath{\overset{#1}{\underset{#2}{\gtrless}}}}
\newcommand{\myvec}[1]{\ensuremath{\begin{pmatrix}#1\end{pmatrix}}}
\newcommand{\mydet}[1]{\ensuremath{\begin{vmatrix}#1\end{vmatrix}}}
\numberwithin{equation}{subsection}
\makeatletter
\@addtoreset{figure}{problem}
\makeatother
\let\StandardTheFigure\thefigure
\let\vec\mathbf
\renewcommand{\thefigure}{\theproblem}
\def\putbox#1#2#3{\makebox[0in][l]{\makebox[#1][l]{}\raisebox{\baselineskip}[0in][0in]{\raisebox{#2}[0in][0in]{#3}}}}
     \def\rightbox#1{\makebox[0in][r]{#1}}
     \def\centbox#1{\makebox[0in]{#1}}
     \def\topbox#1{\raisebox{-\baselineskip}[0in][0in]{#1}}
     \def\midbox#1{\raisebox{-0.5\baselineskip}[0in][0in]{#1}}
\vspace{3cm}
\title{Assignment 1- Probability and Random Variables}
\author{Songa Kotesh Satvik}
\maketitle
\newpage
\bigskip
\renewcommand{\thefigure}{\theenumi}
\renewcommand{\thetable}{\theenumi}
Download all python codes from 
\begin{lstlisting}
https://github.com/KoteshSatvik/AI1103-Probability_and_Random_Variables/blob/main/Assignment-1/Assignment1.py
\end{lstlisting}
%
and latex-tikz codes from 
%
\begin{lstlisting}
https://github.com/KoteshSatvik/AI1103-Probability_and_Random_Variables/blob/main/Assignment-1/Assignment1.tex
\end{lstlisting}
\section{Problem 4.9}
Let X denote the sum of the numbers obtained when two fair dice are rolled. Find the variance and standard deviation of X.
\section{Solution}
 When two fare dice are rolled. The sum of the numbers obtained can have the values 2, 3, 4, 5, 6, 7, 8, 9, 10, 11, 12.\\
$\pr{X}$ = probability of obtaining X as the sum and let us represent the case when first dice shows the number $x_1$ and the second dice shows the number $x_2$ as $(x_1,x_2)$.\\
Then,\\
$\pr{X=2}$= 1/36 : [(1,1)]\\\
$\pr{X=3}$= 2/36 : [(1,2),(2,1)]\\
$\pr{X=4}$= 3/36 : [(1,3),(2,2),(3,1)]\\
$\pr{X=5}$= 4/36 : [(1,4),(2,3),(3,2),(4,1)]\\
$\pr{X=6}$= 5/36 : [(1,5),(2,4),(3,3),(4,2),(5,1)]\\
$\pr{X=7}$= 6/36 : [(1,6),(2,5),(3,4),(4,3),(5,2),(6,1)]\\
$\pr{X=8}$= 5/36 : [(2,6),(3,5),(4,4),(5,3),(6,2)]\\
$\pr{X=9}$= 4/36 : [(3,6),(4,5),(5,4),(6,3)]\\
$\pr{X=10}$= 3/36 : [(4,6),(5,5),(6,4)]\\
$\pr{X=11}$= 2/36 : [(5,6),(6,5)]\\
$\pr{X=12}$= 1/36 : [(6,6)]\\

\begin{table}[hbt!]
\resizebox{\columnwidth}{!}{
\begin{tabular}{|l|c|c|c|c|c|c|c|c|c|c|c|}
\hline
\multicolumn{1}{|c|}{x} & 2 & 3 & 4 & 5 & 6 & 7 & 8 & 9 & 10 & 11 & 12 \\ \hline
$\pr{X}$                   &$\frac{1}{36}$   &$\frac{2}{36}$   &$\frac{3}{36}$   &$\frac{4}{36}$   &$\frac{5}{36}$   &$\frac{6}{36}$   &$\frac{5}{36}$   &$\frac{4}{36}$   &$\frac{3}{36}$    &$\frac{2}{36}$    &$\frac{1}{36}$    \\ \hline
\end{tabular}
}
\caption{Probability Distribution Table of X}
\label{table:1}
\end{table}
For the above problem,we know that.
\begin{align}
p_x\brak{n} &= 
  \begin{cases}
    0 & \text{if } n \leq 1,\\
    \frac{n-1}{36} & \text{if } 2 \leq n \leq 7,\\
    \frac{13-n}{36} & \text{if } 7 < n \leq 12,\\
    0 & \text{if } n>12.
  \end{cases}
\end{align}
\begin{align}
    &Mean,E(X) \nonumber\\
    & =\sum_{k=1}^{12} (k \times p_x\brak{k})\\
    & = \sum_{k=1}^{6}k\times\frac{1}{36}[k-1] + \sum_{k=7}^{12}k\times\frac{1}{36}[13-k]\\
    & = \frac{1}{36}\left[\sum_{k=1}^{6}k(k-1) + \sum_{k=7-6}^{12-6}(k+6)\times[13-(k+6)]\right]\\
    & = \frac{1}{36}\left[\sum_{k=1}^{6}k(k-1) + \sum_{k=1}^{6}(k+6)\times[13-(k+6)]\right]\\
    &= \frac{1}{36}\sum_{k=1}^{6}\left(k(k-1) + (k+6)(7-k)\right)\\
    &= \frac{1}{36}\sum_{k=1}^{6}\left((k^2- k)+ (7k-k^2+42-6k)\right)\\
    &= \frac{1}{36}\sum_{k=1}^{6}\left( 42\right)\\
    &= \frac{1}{36}\left[ 42 \times 6 \right]
\end{align}
Therefore,
 Mean, $E\brak{X} = 7$
 
\begin{align}
  Variance,\sigma^2 &= E\brak{X-E\brak{X}}^2 \\
    &= E\brak{X^2} - \brak{E\brak{X}}^2
\end{align}
Let us consider $E\brak{X^2}$,
\begin{align}
    & E\brak{X^2} \nonumber\\
    &= \left(\sum_{k=1}^{12}(k^2\times p_x(k))\right)\\
    &= \sum_{k=1}^{6}k^2\times\frac{1}{36}[k-1] + 
    \sum_{k=7}^{12}(k)^2\times[13-k]\\
    &\text{by rearrangement we get}\\
    &= \frac{1}{36}\left[\sum_{k=1}^{6}k^2(k-1) + \sum_{k=7-6}^{12-6}(k+6)^2\times[13-(k+6)]\right]\\
    &= \frac{1}{36}\left[\sum_{k=1}^{6}k^2(k-1) + \sum_{k=1}^{6}(k+6)^2(7-k)\right]\\
    &= \frac{1}{36}\sum_{k=1}^{6}\left(k^2(k-1) + (k^2+36+12k)(7-k)\right)\\
    &= \frac{1}{36}\sum_{k=1}^{6}\left((k^3-k^2) + (-k^3-5k^2+48k+252)\right)\\
    &= \frac{1}{36}\sum_{k=1}^{6}(-6k^2+48k+252)\\
    &= \frac{1}{6}\sum_{k=1}^{6}(-k^2+8k+42)\\
    &= \frac{1}{6}\left[ -\frac{(6)(7)(13)}{6} + 8\times\frac{(6)(7)}{2}+ (42)(6) \right]\\
    &= \frac{1}{6}[-91+168+252]\\
    &= \frac{329}{6}
\end{align}

\begin{align}
    \text{ Variance, }\sigma^2 &=E\brak{X^2} - \brak{E\brak{X}}^2\\
    &= \frac{329}{6}- ((7)^2)\\
    &= \frac{329}{6} - 49\\
   \sigma^2 &= \frac{35}{6}
\end{align}
Therefore,\\
Standard deviation, $\sigma=\sqrt{\frac{35}{6}}$
\end{document}